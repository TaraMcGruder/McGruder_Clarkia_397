% Options for packages loaded elsewhere
\PassOptionsToPackage{unicode}{hyperref}
\PassOptionsToPackage{hyphens}{url}
%
\documentclass[
]{article}
\usepackage{amsmath,amssymb}
\usepackage{iftex}
\ifPDFTeX
  \usepackage[T1]{fontenc}
  \usepackage[utf8]{inputenc}
  \usepackage{textcomp} % provide euro and other symbols
\else % if luatex or xetex
  \usepackage{unicode-math} % this also loads fontspec
  \defaultfontfeatures{Scale=MatchLowercase}
  \defaultfontfeatures[\rmfamily]{Ligatures=TeX,Scale=1}
\fi
\usepackage{lmodern}
\ifPDFTeX\else
  % xetex/luatex font selection
\fi
% Use upquote if available, for straight quotes in verbatim environments
\IfFileExists{upquote.sty}{\usepackage{upquote}}{}
\IfFileExists{microtype.sty}{% use microtype if available
  \usepackage[]{microtype}
  \UseMicrotypeSet[protrusion]{basicmath} % disable protrusion for tt fonts
}{}
\makeatletter
\@ifundefined{KOMAClassName}{% if non-KOMA class
  \IfFileExists{parskip.sty}{%
    \usepackage{parskip}
  }{% else
    \setlength{\parindent}{0pt}
    \setlength{\parskip}{6pt plus 2pt minus 1pt}}
}{% if KOMA class
  \KOMAoptions{parskip=half}}
\makeatother
\usepackage{xcolor}
\usepackage[margin=1in]{geometry}
\usepackage{color}
\usepackage{fancyvrb}
\newcommand{\VerbBar}{|}
\newcommand{\VERB}{\Verb[commandchars=\\\{\}]}
\DefineVerbatimEnvironment{Highlighting}{Verbatim}{commandchars=\\\{\}}
% Add ',fontsize=\small' for more characters per line
\usepackage{framed}
\definecolor{shadecolor}{RGB}{248,248,248}
\newenvironment{Shaded}{\begin{snugshade}}{\end{snugshade}}
\newcommand{\AlertTok}[1]{\textcolor[rgb]{0.94,0.16,0.16}{#1}}
\newcommand{\AnnotationTok}[1]{\textcolor[rgb]{0.56,0.35,0.01}{\textbf{\textit{#1}}}}
\newcommand{\AttributeTok}[1]{\textcolor[rgb]{0.13,0.29,0.53}{#1}}
\newcommand{\BaseNTok}[1]{\textcolor[rgb]{0.00,0.00,0.81}{#1}}
\newcommand{\BuiltInTok}[1]{#1}
\newcommand{\CharTok}[1]{\textcolor[rgb]{0.31,0.60,0.02}{#1}}
\newcommand{\CommentTok}[1]{\textcolor[rgb]{0.56,0.35,0.01}{\textit{#1}}}
\newcommand{\CommentVarTok}[1]{\textcolor[rgb]{0.56,0.35,0.01}{\textbf{\textit{#1}}}}
\newcommand{\ConstantTok}[1]{\textcolor[rgb]{0.56,0.35,0.01}{#1}}
\newcommand{\ControlFlowTok}[1]{\textcolor[rgb]{0.13,0.29,0.53}{\textbf{#1}}}
\newcommand{\DataTypeTok}[1]{\textcolor[rgb]{0.13,0.29,0.53}{#1}}
\newcommand{\DecValTok}[1]{\textcolor[rgb]{0.00,0.00,0.81}{#1}}
\newcommand{\DocumentationTok}[1]{\textcolor[rgb]{0.56,0.35,0.01}{\textbf{\textit{#1}}}}
\newcommand{\ErrorTok}[1]{\textcolor[rgb]{0.64,0.00,0.00}{\textbf{#1}}}
\newcommand{\ExtensionTok}[1]{#1}
\newcommand{\FloatTok}[1]{\textcolor[rgb]{0.00,0.00,0.81}{#1}}
\newcommand{\FunctionTok}[1]{\textcolor[rgb]{0.13,0.29,0.53}{\textbf{#1}}}
\newcommand{\ImportTok}[1]{#1}
\newcommand{\InformationTok}[1]{\textcolor[rgb]{0.56,0.35,0.01}{\textbf{\textit{#1}}}}
\newcommand{\KeywordTok}[1]{\textcolor[rgb]{0.13,0.29,0.53}{\textbf{#1}}}
\newcommand{\NormalTok}[1]{#1}
\newcommand{\OperatorTok}[1]{\textcolor[rgb]{0.81,0.36,0.00}{\textbf{#1}}}
\newcommand{\OtherTok}[1]{\textcolor[rgb]{0.56,0.35,0.01}{#1}}
\newcommand{\PreprocessorTok}[1]{\textcolor[rgb]{0.56,0.35,0.01}{\textit{#1}}}
\newcommand{\RegionMarkerTok}[1]{#1}
\newcommand{\SpecialCharTok}[1]{\textcolor[rgb]{0.81,0.36,0.00}{\textbf{#1}}}
\newcommand{\SpecialStringTok}[1]{\textcolor[rgb]{0.31,0.60,0.02}{#1}}
\newcommand{\StringTok}[1]{\textcolor[rgb]{0.31,0.60,0.02}{#1}}
\newcommand{\VariableTok}[1]{\textcolor[rgb]{0.00,0.00,0.00}{#1}}
\newcommand{\VerbatimStringTok}[1]{\textcolor[rgb]{0.31,0.60,0.02}{#1}}
\newcommand{\WarningTok}[1]{\textcolor[rgb]{0.56,0.35,0.01}{\textbf{\textit{#1}}}}
\usepackage{graphicx}
\makeatletter
\def\maxwidth{\ifdim\Gin@nat@width>\linewidth\linewidth\else\Gin@nat@width\fi}
\def\maxheight{\ifdim\Gin@nat@height>\textheight\textheight\else\Gin@nat@height\fi}
\makeatother
% Scale images if necessary, so that they will not overflow the page
% margins by default, and it is still possible to overwrite the defaults
% using explicit options in \includegraphics[width, height, ...]{}
\setkeys{Gin}{width=\maxwidth,height=\maxheight,keepaspectratio}
% Set default figure placement to htbp
\makeatletter
\def\fps@figure{htbp}
\makeatother
\setlength{\emergencystretch}{3em} % prevent overfull lines
\providecommand{\tightlist}{%
  \setlength{\itemsep}{0pt}\setlength{\parskip}{0pt}}
\setcounter{secnumdepth}{-\maxdimen} % remove section numbering
\ifLuaTeX
  \usepackage{selnolig}  % disable illegal ligatures
\fi
\IfFileExists{bookmark.sty}{\usepackage{bookmark}}{\usepackage{hyperref}}
\IfFileExists{xurl.sty}{\usepackage{xurl}}{} % add URL line breaks if available
\urlstyle{same}
\hypersetup{
  pdftitle={flowersurvey.R},
  pdfauthor={taram},
  hidelinks,
  pdfcreator={LaTeX via pandoc}}

\title{flowersurvey.R}
\author{taram}
\date{2024-02-06}

\begin{document}
\maketitle

\begin{Shaded}
\begin{Highlighting}[]
\CommentTok{\# if you want and have the time, githappy is the step by step guide for starting}
\CommentTok{\# github account}


\CommentTok{\# Tasks for Tara:}
\CommentTok{\# {-} read through code and make sure stuff makes sense}
\CommentTok{\#  {-} when you look through the plots, add annotations/comments of what you are seeing}

\CommentTok{\# {-} check for outliers, for any doy, should be above 300}
    \CommentTok{\# for checking, do View(flowersurvey\_DOY) in console}
    \CommentTok{\# check for any doy \textless{}300 and any time diff (negative)}


\CommentTok{\# {-} check the subjective color enteries for other spellings,}
\CommentTok{\#       color/color or color + color}
\CommentTok{\# {-} check for cell enteries in date columns that have "?"}
\CommentTok{\# {-} then redownload google sheet as csv}
    \CommentTok{\#delete the csv that is in the folder}
    \CommentTok{\# replace the csv with the newly downloaded, rename it the same}
\CommentTok{\# {-} rerun all the code and check to see if anything is funky}


\CommentTok{\# {-} for the color proportion, check with chatgpt or ggplot2,}
\CommentTok{\#     assign certain colors to each unique color in the stacked barplot}

\CommentTok{\# {-} is there a way to check if anther and pollen color differ? binary with "if" statement?}
\CommentTok{\#     {-} is there a way to quantify how they differ?}

\CommentTok{\# {-} are there any statistical analyses that could be used?}

\CommentTok{\# {-} is DOY the best method to use? can also explore the data for time between bud and first flower}


\CommentTok{\#The following code is to start looking through Tara\textquotesingle{}s data}
\CommentTok{\#Information on the data:}
\CommentTok{\# Includes information regarding population, reproductive timing, size @ flowering,}
\CommentTok{\#   and subjective color of petal, anther, and pollen (overall pattern too)}


\CommentTok{\# Clean and explore data from daily flower surveys}
\CommentTok{\# don\textquotesingle{}t know how much we will gain from this BUT a good way to get comfortable}
\CommentTok{\# using R and solving certain data puzzles}

\CommentTok{\# Install and load necessary packages}
\CommentTok{\# my top packages are always dplyr, tidyr, and ggplot2 if I\textquotesingle{}m looking at data}

\CommentTok{\# check if the packages are already installed before attempting to install them}
\ControlFlowTok{if}\NormalTok{ (}\SpecialCharTok{!}\FunctionTok{requireNamespace}\NormalTok{(}\StringTok{"dplyr"}\NormalTok{, }\AttributeTok{quietly =} \ConstantTok{TRUE}\NormalTok{)) }\FunctionTok{install.packages}\NormalTok{(}\StringTok{"dplyr"}\NormalTok{)}
\ControlFlowTok{if}\NormalTok{ (}\SpecialCharTok{!}\FunctionTok{requireNamespace}\NormalTok{(}\StringTok{"tidyr"}\NormalTok{, }\AttributeTok{quietly =} \ConstantTok{TRUE}\NormalTok{)) }\FunctionTok{install.packages}\NormalTok{(}\StringTok{"tidyr"}\NormalTok{)}
\ControlFlowTok{if}\NormalTok{ (}\SpecialCharTok{!}\FunctionTok{requireNamespace}\NormalTok{(}\StringTok{"lubridate"}\NormalTok{, }\AttributeTok{quietly =} \ConstantTok{TRUE}\NormalTok{)) }\FunctionTok{install.packages}\NormalTok{(}\StringTok{"lubridate"}\NormalTok{)}
\ControlFlowTok{if}\NormalTok{ (}\SpecialCharTok{!}\FunctionTok{requireNamespace}\NormalTok{(}\StringTok{"ggplot2"}\NormalTok{, }\AttributeTok{quietly =} \ConstantTok{TRUE}\NormalTok{)) }\FunctionTok{install.packages}\NormalTok{(}\StringTok{"ggplot2"}\NormalTok{)}
\ControlFlowTok{if}\NormalTok{ (}\SpecialCharTok{!}\FunctionTok{requireNamespace}\NormalTok{(}\StringTok{"stringr"}\NormalTok{, }\AttributeTok{quietly =} \ConstantTok{TRUE}\NormalTok{)) }\FunctionTok{install.packages}\NormalTok{(}\StringTok{"stringr"}\NormalTok{)}

\CommentTok{\# manipulate and analyze data frames}
\FunctionTok{library}\NormalTok{(dplyr)  }
\end{Highlighting}
\end{Shaded}

\begin{verbatim}
## Warning: package 'dplyr' was built under R version 4.3.2
\end{verbatim}

\begin{verbatim}
## 
## Attaching package: 'dplyr'
\end{verbatim}

\begin{verbatim}
## The following objects are masked from 'package:stats':
## 
##     filter, lag
\end{verbatim}

\begin{verbatim}
## The following objects are masked from 'package:base':
## 
##     intersect, setdiff, setequal, union
\end{verbatim}

\begin{Shaded}
\begin{Highlighting}[]
\CommentTok{\# reshape and tidy data for analysis}
\FunctionTok{library}\NormalTok{(tidyr)}
\end{Highlighting}
\end{Shaded}

\begin{verbatim}
## Warning: package 'tidyr' was built under R version 4.3.2
\end{verbatim}

\begin{Shaded}
\begin{Highlighting}[]
\CommentTok{\# handling date and time data}
\FunctionTok{library}\NormalTok{(lubridate)}
\end{Highlighting}
\end{Shaded}

\begin{verbatim}
## Warning: package 'lubridate' was built under R version 4.3.2
\end{verbatim}

\begin{verbatim}
## 
## Attaching package: 'lubridate'
\end{verbatim}

\begin{verbatim}
## The following objects are masked from 'package:base':
## 
##     date, intersect, setdiff, union
\end{verbatim}

\begin{Shaded}
\begin{Highlighting}[]
\CommentTok{\# visualizing data}
\FunctionTok{library}\NormalTok{(ggplot2)}
\end{Highlighting}
\end{Shaded}

\begin{verbatim}
## Warning: package 'ggplot2' was built under R version 4.3.2
\end{verbatim}

\begin{Shaded}
\begin{Highlighting}[]
\CommentTok{\# working with string operations, good for text extraction}
\FunctionTok{library}\NormalTok{(stringr)}



\DocumentationTok{\#\#\#\# CLEAN AND CREATE NEW DATA FRAME \#\#\#\#}

\CommentTok{\# Now that you have finished the huge task of entering your data...}
\CommentTok{\# Time to load the data using the relative path/directory on the computer}
\NormalTok{flowersurvey }\OtherTok{\textless{}{-}} \FunctionTok{read.csv}\NormalTok{(}\StringTok{"C:/Users/taram/OneDrive/Desktop/TaraClarkiaFloralData/flowersurvey.csv"}\NormalTok{)}

\CommentTok{\# You used "m/d/y" format, but we need to tell the program to read it as that}
\CommentTok{\# instead of simply a random character variable}

\CommentTok{\# mdy(): function to convert any date column to a Date object}
\NormalTok{flowersurvey}\SpecialCharTok{$}\NormalTok{bud\_emerg\_date }\OtherTok{\textless{}{-}} \FunctionTok{mdy}\NormalTok{(flowersurvey}\SpecialCharTok{$}\NormalTok{bud\_emerg\_date)}
\NormalTok{flowersurvey}\SpecialCharTok{$}\NormalTok{flwr\_date }\OtherTok{\textless{}{-}} \FunctionTok{mdy}\NormalTok{(flowersurvey}\SpecialCharTok{$}\NormalTok{flwr\_date)}
\NormalTok{flowersurvey}\SpecialCharTok{$}\NormalTok{scnd\_flwr\_date }\OtherTok{\textless{}{-}} \FunctionTok{mdy}\NormalTok{(flowersurvey}\SpecialCharTok{$}\NormalTok{scnd\_flwr\_date)}

\CommentTok{\# commented out for now, just cause I don\textquotesingle{}t want to right now}
\CommentTok{\#flowersurvey$third\_flwr\_date \textless{}{-} mdy(flowersurvey$third\_flwr\_date)}


\CommentTok{\# Now we should make a new data frame that has DOY instead of the date}
\CommentTok{\# DOY is January 01 = 0}
\CommentTok{\# Create a new data frame \textquotesingle{}flowersurvey\_DOY\textquotesingle{} based on \textquotesingle{}flowersurvey\textquotesingle{}, saying DOY}
  \CommentTok{\# to signify that this new data frame includes dates as DOY}
\CommentTok{\# Huge assumption here is that we are stating all the plants germinated at the}
  \CommentTok{\# same time, we know they all germinated in two (?) week period.}


\CommentTok{\# With the dplyr package, use \%\textgreater{}\% (pipe) operator to manipulate the data }
  \CommentTok{\# in a bunch of ways at the "same time"}
\CommentTok{\#Problem with using DOY is it restarts each year so we want to distinguish btwn}
  \CommentTok{\#2023 and 2024, when starting in 2024, DOY starts at 365 instead of 0}
\CommentTok{\# To convert DOY to a continuous DOY variable,}
  \CommentTok{\# create a column that includes the year and then "if" statements"}

\NormalTok{flowersurvey\_DOY }\OtherTok{\textless{}{-}}\NormalTok{ flowersurvey }\SpecialCharTok{\%\textgreater{}\%}

\CommentTok{\# Create new columns \textquotesingle{}year\_bud\textquotesingle{} and \textquotesingle{}bud\_doy\textquotesingle{}}
  \CommentTok{\# mutate(newcolumn = ANY FUNCTIONS WITH (old column)) function: create new columns based on existing columns}
  \CommentTok{\# can derive new variables this way.}
\CommentTok{\# the year() function extracts the year from the date column.}
  \FunctionTok{mutate}\NormalTok{(}\AttributeTok{year\_bud =} \FunctionTok{year}\NormalTok{(bud\_emerg\_date),}
         \CommentTok{\# The yday() function converts date to DOY}
         \AttributeTok{bud\_doy =} \FunctionTok{yday}\NormalTok{(bud\_emerg\_date)}
         \CommentTok{\# ifelse() function to add 365 to DOY if the year is 2024}
         \SpecialCharTok{+} \FunctionTok{ifelse}\NormalTok{(year\_bud }\SpecialCharTok{==} \DecValTok{2024}\NormalTok{, }\DecValTok{365}\NormalTok{, }\DecValTok{0}\NormalTok{)) }\SpecialCharTok{\%\textgreater{}\%}
  
\CommentTok{\# Create new columns \textquotesingle{}year\_flwr\textquotesingle{} and \textquotesingle{}flwr\_doy\textquotesingle{}}
  \FunctionTok{mutate}\NormalTok{(}\AttributeTok{year\_flwr =} \FunctionTok{year}\NormalTok{(flwr\_date),}
         \AttributeTok{flwr\_doy =} \FunctionTok{yday}\NormalTok{(flwr\_date) }\SpecialCharTok{+} \FunctionTok{ifelse}\NormalTok{(year\_flwr }\SpecialCharTok{==} \DecValTok{2024}\NormalTok{, }\DecValTok{365}\NormalTok{, }\DecValTok{0}\NormalTok{)) }\SpecialCharTok{\%\textgreater{}\%}
  
\CommentTok{\# Create new columns \textquotesingle{}year\_scnd\_flwr\textquotesingle{} and \textquotesingle{}scnd\_flwr\_doy\textquotesingle{}}
  \FunctionTok{mutate}\NormalTok{(}\AttributeTok{year\_scnd\_flwr =} \FunctionTok{year}\NormalTok{(scnd\_flwr\_date),}
         \AttributeTok{scnd\_flwr\_doy =} \FunctionTok{yday}\NormalTok{(scnd\_flwr\_date) }\SpecialCharTok{+} \FunctionTok{ifelse}\NormalTok{(year\_scnd\_flwr }\SpecialCharTok{==} \DecValTok{2024}\NormalTok{, }\DecValTok{365}\NormalTok{, }\DecValTok{0}\NormalTok{)) }\SpecialCharTok{\%\textgreater{}\%}
  
\CommentTok{\#commented this out for now, can bring it in later}
  \CommentTok{\#mutate(year\_third\_flwr = year(third\_flwr\_date),}
         \CommentTok{\#third\_flwr\_doy = yday(third\_flwr\_date) + ifelse(year\_third\_flwr == 2024, 365, 0)) \%\textgreater{}\%}

\CommentTok{\# rename tag\_name as pop\_fam so we know it is telling us the population and family}
  \FunctionTok{rename}\NormalTok{(}\AttributeTok{pop\_fam =}\NormalTok{ tag\_name) }\SpecialCharTok{\%\textgreater{}\%}
  
\CommentTok{\# mutate() to modify existing column and add a new one with just population}
    \CommentTok{\#extracting words from a string, taking the first word, separated by "\_"}
  \FunctionTok{mutate}\NormalTok{(}\AttributeTok{pop\_name =} \FunctionTok{word}\NormalTok{(pop\_fam, }\DecValTok{1}\NormalTok{, }\AttributeTok{sep =} \StringTok{"\_"}\NormalTok{)) }\SpecialCharTok{\%\textgreater{}\%}
 
\CommentTok{\#censoring out any plants that didn\textquotesingle{}t germinate}
  \CommentTok{\#also do not want plants that didn\textquotesingle{}t have a first date of flowering}
  \FunctionTok{filter}\NormalTok{(exclude }\SpecialCharTok{==} \DecValTok{0}\NormalTok{, }\SpecialCharTok{!}\FunctionTok{is.na}\NormalTok{(flwr\_date)) }\SpecialCharTok{\%\textgreater{}\%} 

\CommentTok{\# select() function chooses specific columns to keep in final data frame,}
  \CommentTok{\# add a "{-}" to get rid of certain columns}
   \FunctionTok{select}\NormalTok{(}\SpecialCharTok{{-}}\FunctionTok{c}\NormalTok{(location,}
\NormalTok{             exclude,}
\NormalTok{            notes,}
\NormalTok{            X,}
\NormalTok{            key),}
\CommentTok{\#getting rid of columns that contain the phrase "date"         }
   \SpecialCharTok{{-}}\FunctionTok{contains}\NormalTok{(}\StringTok{"date"}\NormalTok{),}
\CommentTok{\# check to see if removing date and year can be combined in one line}
    \SpecialCharTok{{-}}\FunctionTok{contains}\NormalTok{(}\StringTok{"year"}\NormalTok{)) }\SpecialCharTok{\%\textgreater{}\%}

\CommentTok{\# create two new columns for the time between bud and first, and first and second flower }
  \FunctionTok{mutate}\NormalTok{(}
    \AttributeTok{time\_fst\_flwr =}\NormalTok{ flwr\_doy }\SpecialCharTok{{-}}\NormalTok{ bud\_doy,}
    \AttributeTok{time\_fst\_scnd\_flwr =}\NormalTok{ scnd\_flwr\_doy }\SpecialCharTok{{-}}\NormalTok{ flwr\_doy)}

\DocumentationTok{\#\#\#\# START VISUALIZING DATA \#\#\#\#}

\CommentTok{\# Create a histogram of flowering time for each population}
\FunctionTok{ggplot}\NormalTok{(}\AttributeTok{data =}\NormalTok{ flowersurvey\_DOY) }\SpecialCharTok{+} 
  \FunctionTok{geom\_histogram}\NormalTok{(}\FunctionTok{aes}\NormalTok{(}\AttributeTok{x =}\NormalTok{ flwr\_doy)) }\SpecialCharTok{+}
  \FunctionTok{facet\_wrap}\NormalTok{(.}\SpecialCharTok{\textasciitilde{}}\NormalTok{pop\_name) }\SpecialCharTok{+}
  \FunctionTok{ylab}\NormalTok{(}\StringTok{"Frequency"}\NormalTok{) }\SpecialCharTok{+}
  \FunctionTok{xlab}\NormalTok{(}\StringTok{"Days to flower"}\NormalTok{)}
\end{Highlighting}
\end{Shaded}

\begin{verbatim}
## `stat_bin()` using `bins = 30`. Pick better value with `binwidth`.
\end{verbatim}

\includegraphics{flowersurvey_files/figure-latex/unnamed-chunk-1-1.pdf}

\begin{Shaded}
\begin{Highlighting}[]
\FunctionTok{summary}\NormalTok{(flowersurvey\_DOY)}
\end{Highlighting}
\end{Shaded}

\begin{verbatim}
##    pop_fam          num_buds_at_emerge ttl_stem_lngth_flwr buds_at_first_flwr
##  Length:284         Min.   :1.000      Min.   : 3.30       Min.   :0.000     
##  Class :character   1st Qu.:1.000      1st Qu.: 8.60       1st Qu.:2.000     
##  Mode  :character   Median :2.000      Median :12.30       Median :3.000     
##                     Mean   :1.922      Mean   :12.59       Mean   :2.879     
##                     3rd Qu.:2.000      3rd Qu.:16.50       3rd Qu.:4.000     
##                     Max.   :5.000      Max.   :27.60       Max.   :7.000     
##                     NA's   :194        NA's   :193         NA's   :193       
##  pollen_color_subj  anther_color_subj  overall_petal_color_subj
##  Length:284         Length:284         Length:284              
##  Class :character   Class :character   Class :character        
##  Mode  :character   Mode  :character   Mode  :character        
##                                                                
##                                                                
##                                                                
##                                                                
##  petal_pat_color     bud_doy         flwr_doy     scnd_flwr_doy
##  Min.   :0.0000   Min.   :  8.0   Min.   :321.0   Min.   :325  
##  1st Qu.:1.0000   1st Qu.:324.0   1st Qu.:339.0   1st Qu.:343  
##  Median :1.0000   Median :331.0   Median :347.0   Median :351  
##  Mean   :0.7848   Mean   :331.4   Mean   :350.6   Mean   :355  
##  3rd Qu.:1.0000   3rd Qu.:345.0   3rd Qu.:361.0   3rd Qu.:367  
##  Max.   :1.0000   Max.   :367.0   Max.   :393.0   Max.   :391  
##  NA's   :205      NA's   :96                      NA's   :124  
##    pop_name         time_fst_flwr    time_fst_scnd_flwr
##  Length:284         Min.   :  6.00   Min.   :-5.000    
##  Class :character   1st Qu.: 15.00   1st Qu.: 3.000    
##  Mode  :character   Median : 17.00   Median : 4.000    
##                     Mean   : 19.24   Mean   : 4.831    
##                     3rd Qu.: 20.00   3rd Qu.: 6.000    
##                     Max.   :385.00   Max.   :34.000    
##                     NA's   :96       NA's   :124
\end{verbatim}

\begin{Shaded}
\begin{Highlighting}[]
\CommentTok{\# plot the data}
\FunctionTok{ggplot}\NormalTok{(flowersurvey\_DOY, }\FunctionTok{aes}\NormalTok{(}\AttributeTok{x =}\NormalTok{ flwr\_doy, }\AttributeTok{y =}\NormalTok{ pop\_name)) }\SpecialCharTok{+}
  \FunctionTok{geom\_boxplot}\NormalTok{()}
\end{Highlighting}
\end{Shaded}

\includegraphics{flowersurvey_files/figure-latex/unnamed-chunk-1-2.pdf}

\begin{Shaded}
\begin{Highlighting}[]
\FunctionTok{ggplot}\NormalTok{() }\SpecialCharTok{+}
  \FunctionTok{geom\_boxplot}\NormalTok{(}\AttributeTok{data =}\NormalTok{ flowersurvey\_DOY, }\FunctionTok{aes}\NormalTok{(}\AttributeTok{x =}\NormalTok{ pop\_name, }
                                           \AttributeTok{y =}\NormalTok{ ttl\_stem\_lngth\_flwr))}
\end{Highlighting}
\end{Shaded}

\begin{verbatim}
## Warning: Removed 193 rows containing non-finite values (`stat_boxplot()`).
\end{verbatim}

\includegraphics{flowersurvey_files/figure-latex/unnamed-chunk-1-3.pdf}

\begin{Shaded}
\begin{Highlighting}[]
\DocumentationTok{\#\#\#\# COLOR PROPORTION DATA AND GRAPHING \#\#\#\#}
\CommentTok{\#what about looking at color stuff, possibly look at the proportion of diff. colors}
\CommentTok{\# in each population?}

\CommentTok{\#let\textquotesingle{}s look at pollen color}
\NormalTok{pollen\_color\_prop }\OtherTok{\textless{}{-}}\NormalTok{ flowersurvey\_DOY }\SpecialCharTok{\%\textgreater{}\%}
  \CommentTok{\#filter out rows in the pollen color with NA or empty strings}
  \FunctionTok{filter}\NormalTok{(}\SpecialCharTok{!}\FunctionTok{is.na}\NormalTok{(pollen\_color\_subj) }\SpecialCharTok{\&}\NormalTok{ pollen\_color\_subj }\SpecialCharTok{!=} \StringTok{""}\NormalTok{) }\SpecialCharTok{\%\textgreater{}\%}

    \CommentTok{\#let\textquotesingle{}s get rid of any weird spacing and capitalization}
  \FunctionTok{mutate}\NormalTok{(}
    \AttributeTok{pollen\_color\_subj =} \FunctionTok{str\_trim}\NormalTok{(}\FunctionTok{tolower}\NormalTok{(pollen\_color\_subj))}
\NormalTok{  ) }\SpecialCharTok{\%\textgreater{}\%}
  
  \CommentTok{\#group by population and color, then calculate the count of unique colors for each group}
  \FunctionTok{group\_by}\NormalTok{(pop\_name, pollen\_color\_subj) }\SpecialCharTok{\%\textgreater{}\%}
  \FunctionTok{summarise}\NormalTok{(}\AttributeTok{count =} \FunctionTok{n}\NormalTok{()) }\SpecialCharTok{\%\textgreater{}\%}
  
  \CommentTok{\#group by population, then calculate total count for each population}
  \FunctionTok{group\_by}\NormalTok{(pop\_name) }\SpecialCharTok{\%\textgreater{}\%}
  \FunctionTok{mutate}\NormalTok{(}\AttributeTok{total\_count =} \FunctionTok{sum}\NormalTok{(count),}
         \CommentTok{\#calculate proportion of each color within each population}
         \AttributeTok{proportion =}\NormalTok{ count }\SpecialCharTok{/}\NormalTok{ total\_count)}
\end{Highlighting}
\end{Shaded}

\begin{verbatim}
## `summarise()` has grouped output by 'pop_name'. You can override using the
## `.groups` argument.
\end{verbatim}

\begin{Shaded}
\begin{Highlighting}[]
\CommentTok{\#let\textquotesingle{}s plot?}
\FunctionTok{ggplot}\NormalTok{(pollen\_color\_prop, }\FunctionTok{aes}\NormalTok{(}\AttributeTok{x =}\NormalTok{ pop\_name, }\AttributeTok{y =}\NormalTok{ proportion, }\AttributeTok{fill =}\NormalTok{ pollen\_color\_subj)) }\SpecialCharTok{+}
  \FunctionTok{geom\_bar}\NormalTok{(}\AttributeTok{stat =} \StringTok{"identity"}\NormalTok{, }\AttributeTok{position =} \StringTok{"stack"}\NormalTok{) }\SpecialCharTok{+} \CommentTok{\#creating a stacked bar plot}
  \FunctionTok{labs}\NormalTok{(}\AttributeTok{title =} \StringTok{"Proportion of Colors in Each Population"}\NormalTok{,}
       \AttributeTok{x =} \StringTok{"Population"}\NormalTok{,}
       \AttributeTok{y =} \StringTok{"Proportion"}\NormalTok{) }\SpecialCharTok{+}
  \FunctionTok{theme\_minimal}\NormalTok{()}
\end{Highlighting}
\end{Shaded}

\includegraphics{flowersurvey_files/figure-latex/unnamed-chunk-1-4.pdf}

\begin{Shaded}
\begin{Highlighting}[]
\CommentTok{\#hmmmmmm looks like purple may have been spelled differently}

\CommentTok{\# what about petal color}

\NormalTok{petal\_color\_prop }\OtherTok{\textless{}{-}}\NormalTok{ flowersurvey\_DOY }\SpecialCharTok{\%\textgreater{}\%}
  \CommentTok{\#filter out rows in the pollen color with NA or empty strings}
  \FunctionTok{filter}\NormalTok{(}\SpecialCharTok{!}\FunctionTok{is.na}\NormalTok{(overall\_petal\_color\_subj) }\SpecialCharTok{\&}\NormalTok{ overall\_petal\_color\_subj }\SpecialCharTok{!=} \StringTok{""}\NormalTok{) }\SpecialCharTok{\%\textgreater{}\%}
  
  \CommentTok{\#let\textquotesingle{}s get rid of any weird spacing and capitalization}
  \FunctionTok{mutate}\NormalTok{(}
    \AttributeTok{overall\_petal\_color\_subj =} \FunctionTok{str\_trim}\NormalTok{(}\FunctionTok{tolower}\NormalTok{(overall\_petal\_color\_subj))}
\NormalTok{  ) }\SpecialCharTok{\%\textgreater{}\%}
  
  \CommentTok{\#group by population and color, then calculate the count of unique colors for each group}
  \FunctionTok{group\_by}\NormalTok{(pop\_name, overall\_petal\_color\_subj) }\SpecialCharTok{\%\textgreater{}\%}
  \FunctionTok{summarise}\NormalTok{(}\AttributeTok{count =} \FunctionTok{n}\NormalTok{()) }\SpecialCharTok{\%\textgreater{}\%}
  
  \CommentTok{\#group by population, then calculate total count for each population}
  \FunctionTok{group\_by}\NormalTok{(pop\_name) }\SpecialCharTok{\%\textgreater{}\%}
  \FunctionTok{mutate}\NormalTok{(}\AttributeTok{total\_count =} \FunctionTok{sum}\NormalTok{(count),}
         \CommentTok{\#calculate proportion of each color within each population}
         \AttributeTok{proportion =}\NormalTok{ count }\SpecialCharTok{/}\NormalTok{ total\_count)}
\end{Highlighting}
\end{Shaded}

\begin{verbatim}
## `summarise()` has grouped output by 'pop_name'. You can override using the
## `.groups` argument.
\end{verbatim}

\begin{Shaded}
\begin{Highlighting}[]
\CommentTok{\#let\textquotesingle{}s plot?}
\FunctionTok{ggplot}\NormalTok{(petal\_color\_prop, }\FunctionTok{aes}\NormalTok{(}\AttributeTok{x =}\NormalTok{ pop\_name, }\AttributeTok{y =}\NormalTok{ proportion, }\AttributeTok{fill =}\NormalTok{ overall\_petal\_color\_subj)) }\SpecialCharTok{+}
  \FunctionTok{geom\_bar}\NormalTok{(}\AttributeTok{stat =} \StringTok{"identity"}\NormalTok{, }\AttributeTok{position =} \StringTok{"stack"}\NormalTok{) }\SpecialCharTok{+} \CommentTok{\#creating a stacked bar plot}
  \FunctionTok{labs}\NormalTok{(}\AttributeTok{title =} \StringTok{"Proportion of Colors in Each Population"}\NormalTok{,}
       \AttributeTok{x =} \StringTok{"Population"}\NormalTok{,}
       \AttributeTok{y =} \StringTok{"Proportion"}\NormalTok{) }\SpecialCharTok{+}
  \FunctionTok{theme\_minimal}\NormalTok{()}
\end{Highlighting}
\end{Shaded}

\includegraphics{flowersurvey_files/figure-latex/unnamed-chunk-1-5.pdf}

\begin{Shaded}
\begin{Highlighting}[]
\DocumentationTok{\#\#\#\# CHECK FOR OUTLIERS \#\#\#\#}
\CommentTok{\# see any outliers of weirdness, should do this for each variable}
\CommentTok{\# anything below DOY 300 is wild because we started the experiment after that}

\CommentTok{\#first let\textquotesingle{}s check the day of first flower}
\NormalTok{selected\_id }\OtherTok{\textless{}{-}}\NormalTok{ flowersurvey\_DOY}\SpecialCharTok{$}\NormalTok{location[}\FunctionTok{which}\NormalTok{(flowersurvey\_DOY}\SpecialCharTok{$}\NormalTok{flwr\_doy }\SpecialCharTok{\textless{}} \DecValTok{300}\NormalTok{)]}

\CommentTok{\# Print the result, this will print the location of weird outliers to recheck}
\FunctionTok{print}\NormalTok{(selected\_id)}
\end{Highlighting}
\end{Shaded}

\begin{verbatim}
## NULL
\end{verbatim}

\begin{Shaded}
\begin{Highlighting}[]
\DocumentationTok{\#\#\#\# ADDING CLIMATIC DATA \#\#\#\#}

\CommentTok{\# Reading in CSV file with climate data}
\NormalTok{climate\_data\_1981\_2022\_ALL }\OtherTok{\textless{}{-}} \FunctionTok{read.csv}\NormalTok{(}\StringTok{"C:/Users/taram/OneDrive/Desktop/TaraClarkiaFloralData/ClimateNA\_pop\_information\_1981{-}2022MSY.csv"}\NormalTok{)}

\CommentTok{\#Reformat CSV file to pull out combined Tave, PPT, CMD, and DD1040 across all months}
\NormalTok{ave\_climate\_data\_1981\_2022 }\OtherTok{\textless{}{-}}\NormalTok{ climate\_data\_1981\_2022\_ALL }\SpecialCharTok{\%\textgreater{}\%}
  \CommentTok{\#select necessary columns}
  \FunctionTok{select}\NormalTok{(}\FunctionTok{c}\NormalTok{(}\StringTok{"Year"}\NormalTok{, }\StringTok{"ID2"}\NormalTok{, }\StringTok{"Latitude"}\NormalTok{, }\StringTok{"Longitude"}\NormalTok{, }\StringTok{"Elevation"}\NormalTok{, }\StringTok{"MAT"}\NormalTok{, }\StringTok{"MAP"}\NormalTok{,}
           \StringTok{"DD1040"}\NormalTok{, }\StringTok{"CMD"}\NormalTok{)) }\SpecialCharTok{\%\textgreater{}\%}
  \FunctionTok{rename}\NormalTok{(}\AttributeTok{pop\_name =}\NormalTok{ ID2) }\SpecialCharTok{\%\textgreater{}\%}
  \FunctionTok{group\_by}\NormalTok{(pop\_name) }\SpecialCharTok{\%\textgreater{}\%}
  \FunctionTok{summarize}\NormalTok{(}
    \AttributeTok{ttl\_years =} \FunctionTok{n\_distinct}\NormalTok{(Year),}
    \AttributeTok{MAT\_ave =} \FunctionTok{mean}\NormalTok{(MAT),}
    \AttributeTok{MAT\_sd =} \FunctionTok{sd}\NormalTok{(MAT),}
    \AttributeTok{MAT\_se =}\NormalTok{ MAT\_sd }\SpecialCharTok{/} \FunctionTok{sqrt}\NormalTok{(ttl\_years }\SpecialCharTok{{-}} \DecValTok{1}\NormalTok{),}
    \AttributeTok{MAP\_ave =} \FunctionTok{mean}\NormalTok{(MAP),}
    \AttributeTok{MAP\_sd =} \FunctionTok{sd}\NormalTok{(MAP),}
    \AttributeTok{MAP\_se =}\NormalTok{ MAP\_sd }\SpecialCharTok{/} \FunctionTok{sqrt}\NormalTok{(ttl\_years }\SpecialCharTok{{-}} \DecValTok{1}\NormalTok{),}
    \AttributeTok{CMD\_ave =} \FunctionTok{mean}\NormalTok{(CMD),}
    \AttributeTok{CMD\_sd =} \FunctionTok{sd}\NormalTok{(CMD),}
    \AttributeTok{CMD\_se =}\NormalTok{ CMD\_sd }\SpecialCharTok{/} \FunctionTok{sqrt}\NormalTok{(ttl\_years }\SpecialCharTok{{-}} \DecValTok{1}\NormalTok{),}
    \AttributeTok{DD1040\_ave =} \FunctionTok{mean}\NormalTok{(DD1040),}
    \AttributeTok{DD1040\_sd =} \FunctionTok{sd}\NormalTok{(DD1040),}
    \AttributeTok{DD1040\_se =}\NormalTok{ DD1040\_sd }\SpecialCharTok{/} \FunctionTok{sqrt}\NormalTok{(ttl\_years }\SpecialCharTok{{-}} \DecValTok{1}\NormalTok{),}
    \AttributeTok{latitude =} \FunctionTok{first}\NormalTok{(Latitude),}
    \AttributeTok{longitude =} \FunctionTok{first}\NormalTok{(Longitude),}
    \AttributeTok{elevation =} \FunctionTok{first}\NormalTok{(Elevation)}
\NormalTok{  ) }\SpecialCharTok{\%\textgreater{}\%}
  \FunctionTok{filter}\NormalTok{(}\SpecialCharTok{!}\NormalTok{(pop\_name }\SpecialCharTok{\%in\%} \FunctionTok{c}\NormalTok{(}\StringTok{"BBL"}\NormalTok{, }\StringTok{"GPS"}\NormalTok{))) }\SpecialCharTok{\%\textgreater{}\%}
  \FunctionTok{select}\NormalTok{(}\SpecialCharTok{{-}}\NormalTok{ttl\_years)}

\CommentTok{\# combine climatic data frame to flower survey data frame}
\NormalTok{flowersurvey\_clim }\OtherTok{\textless{}{-}}\NormalTok{ flowersurvey\_DOY }\SpecialCharTok{\%\textgreater{}\%}
  \CommentTok{\#left join by the two columns that include the pick abbreviation in the two data frames}
  \FunctionTok{left\_join}\NormalTok{(ave\_climate\_data\_1981\_2022, }\AttributeTok{by =} \FunctionTok{c}\NormalTok{(}\StringTok{"pop\_name"} \OtherTok{=} \StringTok{"pop\_name"}\NormalTok{))}
\end{Highlighting}
\end{Shaded}


\end{document}
